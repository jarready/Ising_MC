%!TEX program = xelatex
%!TEX TS-program = xelatex
%!TEX encoding = UTF-8 Unicode

\documentclass[12pt]{article} %这个我就不多说了,头文件
\usepackage{url} %这个我也不多说了
\usepackage{fontspec,xltxtra,xunicode} %最新的mactex都有
\defaultfontfeatures{Mapping=tex-text}
\setromanfont{Heiti SC} %设置中文字体
\XeTeXlinebreaklocale “zh”
\XeTeXlinebreakskip = 0pt plus 1pt minus 0.1pt %文章内中文自动换行,可以自行调节

\newfontfamily{\H}{Songti SC} %设定新的字体快捷命令
\newfontfamily{\E}{Weibei SC} %设定新的字体快捷命令


\addtolength{\textwidth}{2cm}
\addtolength{\hoffset}{-1cm}
\addtolength{\marginparwidth}{-1cm}
\addtolength{\textheight}{2cm}
\addtolength{\voffset}{-1cm}

\usepackage{amsmath}
\usepackage{cases}
\usepackage{listings}
\usepackage{xcolor}
\usepackage{graphicx}
\usepackage{subfigure}
\usepackage{cite}

\title{Ising模型}
\author{丁历杰}
\date{}
\begin{document}
\lstset{
numbers=left,
numberstyle= \tiny,
keywordstyle= \color{ blue!70},commentstyle=\color{red!50!green!50!blue!50},
%frame=shadowbox,
rulesepcolor= \color{ red!20!green!20!blue!20}
}
\maketitle
\section{模型简介}

\section{统计分析}

\section{模拟分析}

\subsection{模拟算法}
\subsubsection{Metropolis}
\subsubsection{Swendsen-Wang}
\cite{Swendsen:1987eq}
\subsubsection{Worff}
\subsubsection{Worm}


\subsection{二级相变}

\subsection{一级相变}


\renewcommand\refname{参考文献}
\begin{thebibliography}{99}
    \bibitem{Swendsen:1987eq}
        Swendsen, R. H. \& Wang, J.-S. Nonuniversal critical dynamics in Monte Carlo simulations. Phys. Rev. Lett. 58, 86–88 (1987).

\end{thebibliography}

\end{document}
